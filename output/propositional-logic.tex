% Propositional Logic
% ID: MATH.1.1
% Domain: Foundations & Preliminaries
% Topic: Mathematical Logic & Proof Techniques

\section*{Axiomatic Set Theory}
\subsection*{Domain} Mathematical Logic & Foundations
\subsection*{Subfield} Set Theory

\subsection*{Definition}
Axiomatic Set Theory is a formulation of set theory in which the properties and behavior of sets are defined through a collection of axioms. These axioms provide a foundational framework for reasoning about sets, allowing the derivation of theorems and definitions without reliance on informal notions.

\subsection*{Core Principles}
\begin{itemize}
  \item Axiom of Extensionality: Two sets are equal if and only if they have the same elements.
  \item Axiom of Pairing: For any two sets, there exists a set that contains exactly those two sets.
  \item Axiom of Union: For any set, there exists a set that contains all elements of the sets contained in it.
  \item Axiom of Power Set: For any set, there exists a set of all its subsets, called the power set.
  \item Axiom of Infinity: There exists a set that contains the empty set and is closed under the operation of taking the successor.
\end{itemize}

\subsection*{Key Formulas or Symbolic Representations}
\begin{align*}
  & A = B \iff \forall x \, (x \in A \iff x \in B) \\
  & \{a, b\} = \{x \,|\, x = a \lor x = b\} \\
  & \bigcup A = \{x \,|\, \exists S \in A \, (x \in S)\} \\
  & \mathcal{P}(A) = \{B \,|\, B \subseteq A\}
\end{align*}

\subsection*{Worked Example}
Consider the set \( A = \{1, 2\} \). According to the Axiom of Pairing, we can assert that there exists a set \( B \) such that \( B = \{1, 2\} \). Furthermore, the Axiom of Union implies that the union of the set containing \( A \), denoted \( C = \{A\} \), results in \( \bigcup C = \{1, 2\} \).

\subsection*{Common Pitfalls}
- Confusing equality of sets with equality of their elements; sets are equal based on their elements only, not their descriptions.
- Misunderstanding the difference between finite and infinite sets; axioms often deal with the latter in terms of existence and construction.
- Neglecting the importance of axioms; learners may assume that informal intuitive notions of sets hold universally without verification through axioms.

\subsection*{Connections}
Axiomatic Set Theory serves as the foundation for many other areas in mathematics, including topology, logic, and mathematical analysis. It is often prerequisite knowledge for advanced topics such as Model Theory and Category Theory. Additionally, it connects intimately with the idea of formal systems in Mathematical Logic.

\subsection*{Further Reading}
- Zermelo, E. (1930). "Über Grenzzahlen und die Gleichformigkeit der Beziehungen zwischen den Mengen." 
- Cohen, P. J. (1963). "The Independence of the Continuum Hypothesis."
- Kunen, K. (1980). "Set Theory: An Introduction to Independence."