% Venn Diagrams
% ID: MATH.1.2
% Domain: Foundations & Preliminaries
% Topic: Set Theory & Foundations

\section*{Axiomatic Set Theory}
\subsection*{Domain} Mathematical Logic & Foundations
\subsection*{Subfield} Set Theory

\subsection*{Definition}
Axiomatic set theory is a formalized framework within which sets are defined and manipulated based on a specified set of axioms. Notably, it seeks to establish a foundation for mathematics by articulating the properties of sets without reliance on intuitive notions.

\subsection*{Core Principles}
\begin{itemize}
  \item **Zermelo-Fraenkel Axioms (ZF)**: A collection of axioms that forms the basis for modern set theory.
  \item **Axiom of Extensionality**: Two sets are equal if and only if they have the same elements.
  \item **Axiom of Pairing**: For any sets \(a\) and \(b\), there exists a set that contains exactly \(a\) and \(b\).
  \item **Axiom of Union**: For any set \(A\), there exists a set that contains all elements of the sets that are members of \(A\).
  \item **Axiom of Infinity**: There exists a set that contains the empty set and is closed under the operation of taking the union with a singleton.
\end{itemize}

\subsection*{Key Formulas or Symbolic Representations}
Axiomatically, we may express key relations as follows:

\begin{align*}
    & \text{If } a = b \text{, then } \forall x (x \in a \iff x \in b. \\
    & \text{Let } a, b \text{ be sets, then } \exists c \forall x (x \in c \iff x = a \lor x = b). \\
\end{align*}

\subsection*{Worked Example}
Consider the sets \(A = \{1, 2\}\) and \(B = \{2, 1\}\). According to the Axiom of Extensionality, we conclude that \(A = B\) because they consist of the same elements, namely \(1\) and \(2\).

\subsection*{Common Pitfalls}
Learners may often confuse the equality of sets with the equality of their elements. A common error is asserting that \(A = \{1, 2\}\) and \(B = \{2, 1\}\) are different, neglecting the axiom of extensionality.

\subsection*{Connections}
Axiomatic set theory serves as a building block for many other mathematical areas, including model theory, topology, and analysis. It requires a comprehension of logical notation and elementary proof techniques, establishing a framework for discussing the existence and properties of mathematical entities.

\subsection*{Further Reading}
For a deeper understanding of axiomatic set theory, refer to "Set Theory" by Thomas Jech and "Naive Set Theory" by Paul Halmos. Additionally, foundational texts such as "Mathematics: Form and Function" by Harold Edwards may provide valuable insights into the significance of the axiomatic approach.