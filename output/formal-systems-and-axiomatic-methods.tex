% Formal Systems and Axiomatic Methods
% ID: MATH.1.1
% Domain: Foundations & Preliminaries
% Topic: Mathematical Logic & Proof Techniques

\section*{Axiomatic Set Theory}
\subsection*{Domain} Mathematical Logic & Foundations  
\subsection*{Subfield} Set Theory  

\subsection*{Definition}  
Axiomatic Set Theory is a formal mathematical framework that defines sets and their properties through axioms rather than through explicit construction. Notable axiomatic systems include Zermelo-Fraenkel Set Theory (ZF) and ZF with the Axiom of Choice (ZFC). The axioms delineate the acceptable operations and relations of sets, providing a foundational structure for further mathematical reasoning.

\subsection*{Core Principles}  
\begin{itemize}  
  \item Axiom of Extensionality: Two sets are equal if they have the same elements.  
  \item Axiom of Regularity: Every non-empty set has an element that is disjoint from it.  
  \item Axiom of Pairing: For any two sets, there exists a set that contains exactly those two sets.  
  \item Axiom of Union: For any set, there exists a set that contains exactly the elements of the elements of that set.  
  \item Axiom of Power Set: For any set, there exists a set of all its subsets.  
\end{itemize}  

\subsection*{Key Formulas or Symbolic Representations}  
\begin{align*}  
  & S = \{x \in X \mid P(x)\} \quad \text{(Set defined by property \( P \))} \\  
  & \mathcal{P}(X) \quad \text{(Power set of \( X \))}  
\end{align*}  

\subsection*{Worked Example}  
Consider the sets \( A = \{1, 2\} \) and \( B = \{2, 3\} \). According to the Axiom of Pairing, the set \( C = \{A, B\} \) exists and contains exactly the two sets \( A \) and \( B \). Thus, \( C = \{\{1, 2\}, \{2, 3\}\} \).

\subsection*{Common Pitfalls}  
- Misunderstanding the distinction between sets and their elements.  
- Confusing the Axiom of Choice with the existence of specific sets.  
- Assuming sets can contain themselves, violating the Axiom of Regularity.  

\subsection*{Connections}  
Axiomatic Set Theory underpins various fields within mathematics, including topology, analysis, and algebra. It is essential for understanding higher-order logic and foundational issues, such as the foundations of mathematics and the philosophy of set theory.

\subsection*{Further Reading}  
For foundational texts, consider "Set Theory: An Introduction to Independence" by Kenneth Kunen and "Naive Set Theory" by Paul R. Halmos. For historical context and philosophical discussions, "What is Mathematics?" by Richard Courant and Herbert Robbins is beneficial.