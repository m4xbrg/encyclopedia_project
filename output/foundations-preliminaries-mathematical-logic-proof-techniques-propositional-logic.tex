
\documentclass[12pt]{article}
\usepackage[utf8]{inputenc}
\usepackage{amsmath, amssymb}
\usepackage{geometry}
\usepackage{titlesec}
\usepackage{hyperref}
\geometry{margin=1in}

\titleformat{\section}[block]{\large\bfseries}{}{0em}{}
\titleformat{\subsection}[block]{\normalsize\bfseries}{}{0em}{}

\begin{document}

% Title: Mathematical Logic & Proof Techniques
% ID: MATH.1.1
% Domain: Foundations & Preliminaries
% Topic: Mathematical Logic & Proof Techniques

\textbackslash\{\}section*\{Axiomatic Set Theory\}
\textbackslash\{\}subsection*\{Domain\} Mathematical Logic \textbackslash\{\}& Foundations
\textbackslash\{\}subsection*\{Subfield\} Set Theory

\textbackslash\{\}subsection*\{Definition\}
Axiomatic Set Theory is a foundational framework in mathematics that formulates the principles of set membership and operations using a collection of axioms. The most widely accepted system is Zermelo-Fraenkel Set Theory (ZF), often supplemented with the Axiom of Choice (ZFC), which formalizes how sets are constructed and manipulated.

\textbackslash\{\}subsection*\{Core Principles\}
\textbackslash\{\}begin\{itemize\}
  \textbackslash\{\}item The Axiom of Extensionality: Two sets are equal if and only if they have the same elements.
  \textbackslash\{\}item The Axiom of Regularity: Every non-empty set has a member that is disjoint from it.
  \textbackslash\{\}item The Axiom of Power Set: For any set, there exists a set of all its subsets.
  \textbackslash\{\}item The Axiom of Union: For any set, there exists a set that contains all the elements of the subsets of the original set.
\textbackslash\{\}end\{itemize\}

\textbackslash\{\}subsection*\{Key Formulas or Symbolic Representations\}
\textbackslash\{\}begin\{align*\}
  \textbackslash\{\}& A = B \textbackslash\{\}iff \textbackslash\{\}forall x (x \textbackslash\{\}in A \textbackslash\{\}iff x \textbackslash\{\}in B) \textbackslash\{\}\textbackslash\{\}
  \textbackslash\{\}& P(A) = \textbackslash\{\}\{ B \textbackslash\{\}mid B \textbackslash\{\}subseteq A \textbackslash\{\}\} \textbackslash\{\}\textbackslash\{\}
  \textbackslash\{\}& \textbackslash\{\}bigcup A = \textbackslash\{\}\{ x \textbackslash\{\}mid \textbackslash\{\}exists B \textbackslash\{\}in A (x \textbackslash\{\}in B) \textbackslash\{\}\}
\textbackslash\{\}end\{align*\}

\textbackslash\{\}subsection*\{Worked Example\}
Consider the set \textbackslash\{\}( A = \textbackslash\{\}\{1, 2, 3\textbackslash\{\}\} \textbackslash\{\}). The power set \textbackslash\{\}( P(A) \textbackslash\{\}) consists of all subsets of \textbackslash\{\}( A \textbackslash\{\}):
\textbackslash\{\}[
P(A) = \textbackslash\{\}\{\textbackslash\{\}emptyset, \textbackslash\{\}\{1\textbackslash\{\}\}, \textbackslash\{\}\{2\textbackslash\{\}\}, \textbackslash\{\}\{3\textbackslash\{\}\}, \textbackslash\{\}\{1, 2\textbackslash\{\}\}, \textbackslash\{\}\{1, 3\textbackslash\{\}\}, \textbackslash\{\}\{2, 3\textbackslash\{\}\}, \textbackslash\{\}\{1, 2, 3\textbackslash\{\}\}\textbackslash\{\}\}
\textbackslash\{\}]
This construction illustrates the Axiom of Power Set.

\textbackslash\{\}subsection*\{Common Pitfalls\}
Learners often confuse the concepts of a set and its elements, mistakenly believing that a set can contain itself as an element, which is prohibited in standard set theory.

\textbackslash\{\}subsection*\{Connections\}
Axiomatic Set Theory operates as the foundation for various branches of mathematics, notably in areas such as topology, algebra, and analysis, where the concept of a set underpins the structure of mathematical objects.

\textbackslash\{\}subsection*\{Further Reading\}
For comprehensive coverage, see:
- Enderton, H. B. (1977). \textbackslash\{\}textit\{Elements of Set Theory\}. Academic Press.
- Jech, T. J. (2003). \textbackslash\{\}textit\{Set Theory: The Third Millennium Edition, Revised and Expanded\}. Springer.

\end{document}
